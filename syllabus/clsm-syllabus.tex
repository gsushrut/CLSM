\documentclass[12pt]{mcplain}
\setlinespacing{0}
\setparspacing{1}
\setparindentation{0}
\thispagestyle{empty}
\geometry{vmargin=1in,hmargin=1in}

\usepackage{hyperref}
\usepackage{multicol}

\usepackage{times}  % for PostScript fonts, needed to mix it and bf for sf

%% \usepackage[draft]{draftmark}
%% \draftmarksetup{mark=PRELIMINARY,scale=0.6,fontfamily=phv}




\sloppy

\usepackage{probset}
\usepackage{syllabus}
\usepackage{isotope}

% watermark
% \usepackage{background}
% \backgroundsetup{contents={PRELIMINARY},color=black,opacity=0.15}

\begin{document}


%%%%%%%%%%%%%%%%%%%%%%%%%%%%%%%%%%%%%%%%%%%%%%%%%%%%%%%%%%%%%%%%
% title block
%%%%%%%%%%%%%%%%%%%%%%%%%%%%%%%%%%%%%%%%%%%%%%%%%%%%%%%%%%%%%%%%

\syllabustitle{Computational Laboratory in Statistical Mechanics}
\syllabussubtitle{PHYS ????? / Spring 2020 / 2 credits}

\begin{multicols}{3}

Sushrut Ghonge\\
Office TBD\\
sghonge@nd.edu

\columnbreak
Kevin Howard\\
103 Nieuwland Science Hall\\
khoward5@nd.edu

\columnbreak

Prof. Mark Caprio\\
209 Nieuwland Science Hall\\
mcaprio@nd.edu
\end{multicols}

%%%%%%%%%%%%%%%%%%%%%%%%%%%%%%%%%%%%%%%%%%%%%%%%%%%%%%%%%%%%%%%%
% logistics block
%%%%%%%%%%%%%%%%%%%%%%%%%%%%%%%%%%%%%%%%%%%%%%%%%%%%%%%%%%%%%%%%

\syllabusseparator

\syllabussection{Laboratory meetings}

\begin{leftindent}
 Location/Time:TBD %Tuesday / Thursday, 2:00 PM -- 3:15 PM (Location: Jordan Hall of Science, Room 412)
\end{leftindent}

\syllabusseparatorpart{Description}

\syllabussection{Why this course?}
%
% The physics problems you encounter in most courses can be solved with paper and
% pencil. Physics problems in the real world, however, whether in research or
% applications, typically cannot. While paper and pencil are still needed for
% these problems, at some stage in the process it becomes essential to make use of
% numerical methods of solution, and therefore computation.
%
% In the introductory-level Computational Methods in Physics course, you acquired
% a basic foundation in the knowledge and skills needed for computational problem
% solving.  You are now prepared to bring your newfound computational abilities to
% your upper level courses.
%
% The present course, \textit{Computational Laboratory in Quantum Mechanics}, is a
% computational laboratory companion to Quantum Mechanics~II.  You will hone your
% computational skills by tackling realistic applications based on the physical
% concepts you learn in your courses.  Conversely, you will use your computational
% explorations to get a deeper, hands-on understanding of these same physical
% concepts.



\syllabussection{Course structure}


This is a project-based course, built around independent readings, group
meetings, and semi-independent work in loose collaboration with lab partners.
The meetings will be very much in the spirit of research group meetings.  They
will include: seminar-style discussion of readings in preparation for each
project, brainstorming and outlining approaches for accomplishing the goals of
the project, status updates and exchange of ideas as challenges arise, and
sharing of results and insights after each project is completed.  A portion of
each session will be open time dedicated to working on the current project, in a
collaborative setting.  Scheduling will be flexible, as we determine the most
efficient scheme for working on each stage of the project; in initial stages of
a project, we will often spend more meeting time discussing and planning, while
in later stages we will devote more time to development.

\clearpage

\syllabussection{What you have learned so far}

From Computational Methods in Physics, you acquired a basic foundation in the
three fundamental elements of computational problem solving:

(1)~\textit{Programming}. You are now able to speak the language of the
computer.  You are able to plan, code, and debug a programming project
systematically and with confidence.  You are now accustomed to writing programs
which don't merely ``work'' but which are well-structured, divided into logical
subunits, and clearly documented.  Your programs are therefore straightforward
to debug, and you write your code in a general way so that you can easily reuse
and extend it as your scientific goals expand.

(2)~\textit{Numerical methods}.  You have also developed proficiency with many
of the numerical methods used for doing the calculations in scientific problems.
Your initial introduction focused on rootfinding, numerical integration,
integration of differential equations, evaluation of special functions, and an
introduction to numerical linear algebra.  However, more important than the
\textit{specific} numerical methods you learned in Computational Methods in
Physics is the fact that you now have a general sense of the \textit{use and
limitations} of numerical methods.  You have demonstrated that you can read
about a new numerical algorithm on your own, implement code for this algorithm,
and investigate the numerical errors (algorithmic and roundoff) and stability of
the algorithm.

(3)~\textit{Translation}. The least tangible, but arguably most important,
aspect of solving a problem computationally is the step of translation. That is,
how do we combine the various ingredients we learned above, to solve an actual
problem? We must take a general physical framework (such as Newtonian mechanics
or the Schr\"odinger equation) and translate this into a concrete set of
algorithms and calculations, to be carried out numerically, for the given
problem. Which equation or equations is most suitable for solution with the
computer, and how can we best set about doing this? Computational problem
solving, as with any problem solving, must largely be learned from examples and
experience.  While you gained some initial exposure in Computational Methods in
Physics, translation will be the focus of the Computational Laboratory in
Statistical Mechanics.

\syllabussection{What we will learn in this laboratory}

\begin{leftindent}[0.5in]
 \footnotesize
 Computation is an integral part of modern science, and the ability to exploit
 effectively the power offered by computers is therefore essential to a working
 physicist.  The proper application of a computer to modeling physical systems
 is far more than blind ``number crunching'', and the successful computational
 physicist draws on a balanced mix of analytically soluble examples, physical
 intuition, and numerical work to solve problems that are otherwise
 intractable.~---~\textit{Steven E.~Koonin}
\end{leftindent}
%
% We will move beyond the simple textbook case of the hydrogen atom, to see how
% quantum mechanical methods are applied for realistic problems in atomic,
% nuclear, or condensed matter physics. Some of the problems in this course will
% help you to refine your understanding of the basic principles of quantum
% mechanics (\textit{e.g.}, by studying the propagation of wave packets).  Others
% will allow you to move beyond the few toy problems which can be solved
% analytically, to more realistic problems of the type which occur in actual
% research.  You will thereby develop a feel for how quantum mechanics is applied
% in practice.  Indeed, many of the problem-solving approaches you learn in
% Quantum Mechanics~II (such as the variational principle or the WKB
% approximation) only come into their own (or even make any practical sense) when
% they are applied numerically, as they will be in this computational laboratory.
%
% As a byproduct, you will continue to develop your proficiency in scientific
% programming and numerical methods.  Depending on the choice of projects, you
% will explore solution methods for partial differential equations and boundary
% value problems, the computation of Fourier transforms, and the methods of
% numerical linear algebra.  You will also have opportunities to further your
% knowledge of programming methods (\textit{e.g.}, data structures and object
% oriented programming) and to explore methods of graphical visualization.

\syllabussection{Prerequisites/corequisites}


%% \begin{leftindent}
%% \textsf{Prerequisite:} Computational
%% Methods in Physics (PHYS~20420)
%% \\
%% \textsf{Prerequisite/corequisite:} Quantum Mechanics II (PHYS~40454)
%% \end{leftindent}

% \begin{leftindent}
%  Computational
%  Methods in Physics (PHYS~20420) must have been completed previously;
%  Quantum Mechanics II (PHYS~40454) may be taken concurrently or may have been taken previously.
% \end{leftindent}



\syllabusseparatorpart{Organization}

\syllabussection{Resources}
%
% Our aim is to understand both the quantum mechanics and the computational
% approaches in depth, at a level which prepares you for research-grade work.  The
% laboratory projects will be at roughly the level of and in the same spirit as
% those in Koonin's Caltech computational physics course.  Our principal resources
% will be:
%
% \begin{leftindentlist*}{\baselineskip}
%   \item Paolo Giannozzi, \textit{Numerical Methods in Quantum Mechanics},
%   lecture notes, University of Udine (2016).
%
%   \item Morten Hjorth-Jensen, \textit{Computational Physics}, lecture notes,
%   University of Oslo (2015).
%  \url{https://github.com/CompPhysics/ComputationalPhysics/tree/master/doc/Lectures}
% \end{leftindentlist*}
%
% We will also draw upon several quantum mechanics textbooks, especially (on reserve):
%
% \begin{leftindentlist*}{\baselineskip}
%
%   \item Mark Newman, \textit{Computational Physics} (Independently published) (2012)
%
%   \item John S.~Townsend, \textit{A Modern Approach to Quantum Mechanics} (University
%   Science Books, Herndon, VA, 2012).
%
%   \item Ramamurti Shankar, \textit{Principles of Quantum Mechanics} (Springer,
%   New York, NY, 1994).
%
%   \item Steven E.~Koonin and Dawn C.~Meredith, \textit{Computational
%   Physics: FORTRAN Version} (Addison-Wesley, Redwood City, CA, 1990).
% \end{leftindentlist*}

%% We will also continue to draw upon the textbook you have used in
%% Computational Methods in Physics:
%%
%% \begin{leftindentlist*}{\baselineskip}
%% \item
%% Mark Newman, \textit{Computational Physics}, revised ed.\
%% (CreateSpace, 2012).
%% \end{leftindentlist*}
%
% Further more specialized readings for the projects may be taken from
% textbooks, pedagogical papers in the \textit{American Journal of
% Physics} (http://ojps.aip.org/ajp), original research papers, or
% online resources.

\syllabussection{Software}

This course will be based on Python 3, the language you used in Computational
Methods in Physics. We will also make use of some of the well-developed
scientific libraries for Python: the numerical linear algebra library NumPy, the
scientific library SciPy, and the plotting library matplotlib. You will need
to have these installed on your laptop.

Note: In practice, if you wish to write your code in Python 2.7, this will be permissible. There are certain features in Python 3 which we will often take for granted in this class (division of integers being a prime example) which you will be responsible for accounting for in your submitted code. Do so at your own risk! Further, one should always include the language version utilized for a project within the document header.

\syllabussection{Projects}

This course will be structured around projects, each of several weeks' duration.
The life-cycle of a project will be as follows:

\begin{leftindent}

  (1)~Readings on the physical problem and any relevant numerical methods
  \textbf{will be required}, with reading assessments as appropriate.  Since
  the basic physical principles will typically have already been covered in
  Quantum Mechanics~I or~II, the readings will focus on the specific problem at
  hand and the computational details.

  (2)~A guided discussion of the reading will provide opportunities for
  clarification and questions.  This will be a seminar-style discussion~--- you
  will have opportunities to present and explain portions of the reading to your
  peers and/or raise questions on the parts which are still not clear to you.
  Discussion will focus on putting this physical problem in context and on
  planning an approach for carrying out the project.  Getting to the ``correct
  answer'' is not the end goal, but just the start!  What \textit{insight} can
  we gain about the physical system?  How does its behavior depend upon the
  parameters of the problem?

  An important step here will be reviewing or working out \textit{analytic}
  solutions to test cases which we will use to benchmark our codes.  We will
  also usually need to transform the mathematical formulation problem to a form
  which is more amenable to numerical solution (\textit{e.g.}, transformation to
  convenient units, change of variable).

  (3)~You will continue to work on carrying out the project. Planning and coding
  are the ``obvious'' parts.  But this is research we are doing.  Just as
  important is \textit{testing and validation}~--- your code will not give the
  correct answer the first time!  How do you know when you can finally believe
  your results?  (Can we reproduce the analytical results we derived ahead of
  time?)

  Based on the success of the partner system in Computational Methods in
  Physics, you will continue to work in loose collaboration with a partner, for
  mutual support in planning, programming, and carrying out the physics studies.
  A new partner will be assigned for each project.  This has been found to be
  necessary both to keep the working relationship fresh and to ensure full
  exchange of the lessons and ideas learned across the course of the semester.

  (4)~You will turn in a  project report will consist of the code, results
  (output and figures), and a report in the form of a paper exploring the
  computational issues, validation procedure, and physical insights. Part of our
  group discussions will be devoted to comparing our results, determining what
  more might be needed in the way of validation and numerical or physical
  exploration, and planning the contents of this paper.

  You will be expected to develop code collaboratively. \textit{This will
  require writing code which is structured sufficiently well to allow multiple
  people to contribute at the same time.} However, you will write your project
  reports separately, and \textit{your project reports will be separately
  evaluated on the quality of work and understanding they communicate}.

  \end{leftindent}

\syllabussection{Grading}

Grades will be assigned based on the following aspects:

\begin{leftindentlist*}{\baselineskip}

 \item \textbf{Reading Quizzes} (10\%) -- completion of required readings before
 classes;

 \item \textbf{Discussion/Participation} (15\%) -- active and engaged
 particpation during in-class discussion and development time;

 \item \textbf{Projects} (25\% each) -- project reports and code, with rubric
 distributed later.

\end{leftindentlist*}

\syllabussection{Topics}

%
% \textit{Previous years' projects have included:}
%
% \begin{leftindentlist*}{\baselineskip}
%
%  \item Solution of the time-independent Schr\"odinger equation in one
%  dimension (shooting method).  \textit{Test cases:} square well, harmonic oscillator.
%  \textit{Application:} quartic (anharmonic) oscillator.
%
%  \item Solution of the time-independent Schr\"odinger equation~---
%  %% boundary value problems in one dimension,
%  matrix solution by representation in a basis or on a discrete mesh, variational
%  methods;
%  \textit{Exploration:}  double-well potentials
%
%  \item Realistic solution of time-independent Schr\"odinger equation
%  with spherical symmetry.  \textit{Test case:} Coulomb (hydrogenic)
%  problem. \textit{Applications:} nuclear Woods-Saxon potential, atomic
%  screening potential for \isotope{Na}.
%
%  \item Solution of the two-electron problem in the Hartree approximation
%  \textit{Test case:} \isotope{He} atom. \textit{Application:} \isotope{He}-like (two-electron) ions with $Z>2$.
%
%  \item Scattering phase shifts with spherical symmetry.
%  \textit{Test case:} hard sphere.
%  \textit{Application:} atomic scattering with the Lennard-Jones potential.
% \end{leftindentlist*}
%
% \textit{However, we have considerable freedom to choose from a diverse range of topics.  For instance:}
%
% \begin{leftindentlist*}{\baselineskip}
%   \item Solution of the time-dependent Schr\"odinger equation~--- direct
%   solution (instabilities, leapfrog method, implicit methods), spectral methods
%   (and the Fourier transform);
%   \textit{Exploration:} scattering of wave packets off barriers and
%   wells, resonance phenomena
%
%   \item Further studies in solution methods~--- configuration interaction
%   calculations for indistinguishable particles, Quantum Monte Carlo methods,
%   Schr\"odinger equation in momentum space (bound states and scattering)
%
%   \item Applications in atomic and nuclear physics~--- solution for
%   $\isotope{He}$ wave functions and energies (variational or Hartree-Fock
%   methods),
%   %% screening potential in alkalai atoms,
%   fine structure in $\isotope{Na}$ atoms (numerical
%   integration), $\isotope{H}$ molecular ion, Thomas-Fermi model, vacuum
%   polarization
%
%   \item Applications in condensed matter physics~--- exact diagonalization of
%   quantum spin models (Heisenberg model), electrons in a periodic potential
%   (band structure of $\isotope{Si}$)
%
%
% \end{leftindentlist*}



{\enlargethispage{20ex}\mbox{}\hfill{\scriptsize\today}}

\end{document}
